\rhead[\thepage]{\scriptsize{CAPÍTULO \thechapter}. \rightmark}
\lhead[CAPÍTULO \thechapter. \leftmark]{}
%======================================================================
\chapter{Lógica matemática}
\label{LM}
\markboth{Lógica matemática}{Lógica matemática}
%======================================================================


El estudio de la lógica proporciona las herramientas necesarias para evaluar la validez de un argumento. Este marco es aplicable principalmente en las matemáticas pero se puede extrapolar a otras areas del conocimiento, permitiendo un razonamiento preciso y un uso adecuado del lenguaje matemático. Así, se facilita el empleo de simbolismos que nos permitan razonar de manera válida mediante reglas establecidas. En esta disciplina, se trabaja con elementos fundamentales llamados proposiciones..\\

\begin{mydef}
\textbf{Proposición}. Es una expresión con sentido en un lenguaje, que afirma o niega algo y proporciona una información. Usaremos el término proposición para designar una expresión de la cual tenga sentido inequívoco decir verdadera o falsa en un cierto contexto. Se simbolizan con las letras minúsculas p, q, r, s, etc.
\end{mydef}

Las proposiciones tienen \textit{valores de verdad} que pueden ser verdadero (V) o falso (F). Además, las proposiciones se pueden conectar entre si con un \textit{conectivo lógico} para formar nuevas proposiciones.\\

\section{Conectivos lógicos}

\begin{mydef}
\textbf{Conectivo lógico}. Es un elemento de la lógica matemática que permite obtener nuevas proposiciones a partir de proposiciones ya entregadas. Los conectivos son varios, como por ejemplo: no, y, o, entonces, si y solo sí, etc.
\end{mydef}

Al momento de utilizar conectivos lógicos para unir proposiciones, estas se puede clasificar en proposiciones lógicas simples o compuestas.

\begin{mydef}
\textbf{Proposición simple}. Una proposición simple o atómica es la cual no incluye conectivos lógicos.
\end{mydef}

\begin{myexample}
Las siguientes  proposiciones lógicas son simples
\end{myexample}
\noindent\textit{p:} ``El primer trimestre del $2021$ se realizaron $171\hspace{2px}626$ intervenciones quirúrgicas''\\
\textit{q:} ``El primer trimestre del $2022$ se realizaron $197\hspace{2px}494$ intervenciones quirúrgicas''\\

\noindent\textit{p:} ``Los partos en el primer trimestre del $2021$ fueron $24\hspace{2px}383$''\\
\textit{q:} ``Los partos en el primer trimestre del $2022$ fueron $27\hspace{2px}216$''\\

\noindent\textit{p:} ``Las teleconsultas en el primer trimestre del $2021$ fueron $239\hspace{2px}079$''\\
\textit{q:} ``Las teleconsultas en el primer trimestre del $2022$ fueron $118\hspace{2px}216$''\\

Todas las proposiciones anteriores están obtenidas del resumen ejecutivo trimestral 2021-2022 de la subsecretaría de redes asistenciales del Ministerio de salud de Chile.\\

\begin{mydef}
\textbf{Proposición compuesta}. Una proposición compuesta o molecular es la que resulta al combinar proposiciones simples con conectivos lógicos.
\end{mydef}

\begin{myexample}
Las siguientes proposiciones lógicas  son compuestas
\end{myexample}
\noindent\textit{p:} ``El primer trimestre del $2022$ superó en $25\hspace{2px}868$ intervenciones quirúrgicas al $2021$, entonces las intervenciones aumentaron en un $15,07\%$ de una año a otro ''.\\

\noindent\textit{q:} ``El primer trimestre del $2022$ nacieron $2\hspace{2px}833$ bebés más que el mismo periodo del $2021$, entonces los partos aumentaron $11,62\%$''.\\

\noindent\textit{r:} ``Las teleconsultas del $2021$ y las teleconsultas del $2022$ superaron las $100\hspace{2px}000$''.\\

Es momento de traspasar el lenguaje a simbologías. Lo primero es asignar letras ($p$, $q$, $r$, etc) a las preposiciones, ahora representaremos los conectivos lógicos: negación, conjunción, disyunción, condicional y bicondicional.

\subsection{Negación}
Se llama negación de una proposición p, a la proposición \textit{no p}. Su notación es $\sim p$, $-p$ o $p'$. Este conectivo lógico es el único que actúa sobre una sola proposición y cambia el valor de verdad, es decir, si la proposición $p$ es verdadera cambia a falsa y si es falsa cambia a verdadera. \\

Estos valores se pueden representar en la tabla de verdad y por ser el primer caso de conectivo lógico, explicaremos el paso a paso como se construye.\\

Lo primero es identificar cuantas proposiciones se tienen (no se consideran proposiciones adicionales a las que están negadas), por ejemplo, si tengo tres proposiciones y alguna está negadas, solo debo escribir $p$, $q$ y $s$. Luego debajo de las proposiciones se debe escribir todas las combinaciones posibles, como los valores de verdad son solo 2, la tabla crece de la forma $2^{n}$, donde n es el número de preposiciones.  Por lo que para el ejemplo recien dado, serian 3 columnas de 8 filas cada una. Finalmente, escribir en la primera columna todas las operaciones a realizar entre las proposiciones y aplicar las reglas de los conectores para ver el resultado final de la tabla.\\

\begin{table}[h!]
	\begin{center}
		\begin{tabular}{|c|c|}
\hline
$p$&$\sim$ $p$\\
\hline
V&F\\
\hline
F&V\\
\hline
		\end{tabular}
	\end{center}
\caption[Tabla de verdad del operador lógico negación.]{Tabla de verdad del operador lógico negación. Se considera una sola proposición, que es $p$ y no se toma en cuenta $\sim p$. El largo de la tabla hacia abajo es $2^{1}=2$.}
\end{table}

\begin{myexample}
Sea $p$ una proposición simple y $\sim p$ su negación.
\end{myexample}
\noindent $p:$ ``Todos los pacientes se mejoran con el tratamiento''.\\
$\sim p:$ ``No todos los pacientes se mejoran con el tratamiento'' o ``Algunos pacientes no se mejoran con el tratamiento''. 

\subsection{Conjunción}
Se llama conjunción cuando dos proposiciones $p$ y $q$ son unidas por el conectivo lógico $y$, resultando una proposición compuesta. Se pronuncia ``$p$ $y$ $q$'' y su notación es $p\wedge q$. La tabla de verdad para dos proposiciones de la conjunción es la siguiente:
\begin{table}[h!]
	\begin{center}
		\begin{tabular}{|c|c|c|}
\hline
$p$&$q$&$p\wedge q$\\
\hline
V&V&V\\
\hline
V&F&F\\
\hline
F&V&F\\
\hline
F&F&F\\
\hline
		\end{tabular}
	\end{center}
\caption[Tabla de verdad del operador lógico conjunción.]{Tabla de verdad del operador conjunción. La proposición $p\wedge q$ es falsa si al menos una de las proposiciones simples, p ó q, es falsa y es verdadera si ambas proposiciones son verdaderas. En este caso se consideran 2 proposiciones, por lo que el largo de la tabla es $2^{2}=4$.}
\end{table}

\begin{myexample}
Sean $p$ y $q$ dos proposiciones simples. Escriba la conjunción entre ambas proposiciones.\\

\noindent\textit{p:} ``El cuadrado tiene cuatro lados''.\\
\textit{q:} ``El cuadrado tiene cuatro ángulos de $90^{o}$''.\\
\textit{$p\wedge q:$} ``El cuadrado tiene cuatro lados y el cuadrado tiene cuatro ángulos de $90^{°}$'' o ``El cuadrado tiene cuatro lados y cuatro ángulos de $90^{o}$''.
\end{myexample}


\subsection{Disyunción inclusiva}
Es la proposición resultante al conectar dos proposiciones simples, p y q, mediante el conector lógico $\vee$. Se lee ``$p$ $o$ $q$'' y es representado por $p \vee q$. La tabla de verdad es la siguiente:
\begin{table}[h!]
	\begin{center}
		\begin{tabular}{|c|c|c|}
\hline
$p$&$q$&$p \vee q$\\
\hline
V&V&V\\
\hline
V&F&V\\
\hline
F&V&V\\
\hline
F&F&F\\
\hline
		\end{tabular}
	\end{center}
\caption[Tabla de verdad del operador lógico disyunción inclusiva]{Tabla de verdad de la disyunción inclusiva. La disyunción inclusiva es falsa solo en el caso cuando ambas proposiciones son falsas, en todos los otros casos es verdadera.}
\end{table}

\begin{myexample}
Sean $p$ y $q$ dos proposiciones simples. Escriba la disyunción inclusiva entre ambas proposiciones.\\

\noindent\textit{p:} ``Ir al Lollapalooza''.\\
\textit{q:} ``Ir a ver a tus primos al campo''.\\
\textit{$p\vee q:$} ``Ir al Lollapalooza o ir a ver a tus primos al campo''.
\end{myexample}


\subsection{Disyunción exclusiva}
Es la proposición compuesta que resulta al conectar dos preposiciones simples mediante el conector lógico de disyunción exclusiva. Se denota como $p\veebar q$ y se lee ``$o$ $p$ $o$ $q$''. La tabla de verdad es la siguiente:\\
\begin{table}[h!]
	\begin{center}
		\begin{tabular}{|c|c|c|}
\hline
$p$&$q$&$p\veebar q$\\
\hline
V&V&F\\
\hline
V&F&V\\
\hline
F&V&V\\
\hline
F&F&F\\
\hline
		\end{tabular}
\caption[Tabla de verdad del operador lógico disyunción exclusiva.]{Tabla de verdad de disyunción exclusiva. Para el caso de este conector lógico es verdadero solo cuando las proposiciones simples tienen valor de verdad contrario.}
	\end{center}
\end{table}
\newpage
\begin{myexample}
Sean $p$ y $q$ dos proposiciones simples. Escriba la disyunción exclusiva entre ambas proposiciones.\\

\noindent\textit{p:} ``El vaso está feo''.\\
\textit{q:} ``La leche está adulterada''.\\
\textit{$p\veebar q:$} ``O el vaso está feo o la leche está adulterada''.\\
\end{myexample}



\noindent $p\veebar q:$ Esta tarde iré al cine o me quedaré en casa estudiando. Disyunción exclusiva.\\

La diferencia entre las disyunciones, es que la inclusiva necesita solamente una proposición verdadera, por eso el nombre. Mientras que la exclusiva solamente es verdadera en el caso de que una proposición lógica es verdadera. Esta ultima se puede presentar en casos donde hay incompatibilidad de las proposiciones, por ejemplo: yo apruebo el ramo o lo repruebo. En el ejemplo no pueden ocurrir ambas cosas, pero para la primera disyunción si puede pasar que ambas proposiciones sean compatibles.
\subsection{Condicional}
Se llama condicional de la proposicion $p$ y $q$ a la proposición ``Si $p$, entonces $q$''. La primera proposición se llama antecedente y la segunda se llama consecuente. Se denota $p\rightarrow q$ y la tabla de verdad es la siguiente:
\begin{table}[h!]
	\begin{center}
		\begin{tabular}{|c|c|c|}
\hline
$p$&$q$&$p\rightarrow q$\\
\hline
V&V&V\\
\hline
V&F&F\\
\hline
F&V&V\\
\hline
F&F&V\\
\hline
		\end{tabular}
	\end{center}
\caption[Tabla de verdad del operador lógico condicional.]{Tabla de verdad del operador lógico condicional. Para el condicional se da un solo caso falso y es cuando la primera proposición (antecedente) es verdadera y la segunda (consecuente) es falsa.}
\end{table}

\begin{myexample}
Sean $p$ y $q$ dos proposiciones simples. Escriba el condicional entre ambas proposiciones.\\

\noindent\textit{p:} ``El núcleo atómico está conformado solo por partículas''.\\
\textit{q:} ``El protón\footnote{Partícula subatómica (que tiene dimensiones menores a las de un átomo) de carga positiva. Es más liviano que neutrón (Sin carga) y más pesado que un electrón (Carga negativa).} es una partícula''.\\
\textit{$p\rightarrow q:$} ``El núcleo atómico está formado solo por partículas, entonces el protón es una partícula.''\\
\end{myexample}


\subsection{Bicondicional}
 Se llama bicondicional de las proposiciones $p$ y $q$ a la proposición ``$p$ si y solo si $q$''. Su notación es $p \longleftrightarrow q$ y su tabla de verdad es la siguiente:\\
\begin{table}[h!]
 	\begin{center}
		\begin{tabular}{|c|c|c|}
\hline
$p$&$q$&$p \longleftrightarrow q$\\
\hline
V&V&V\\
\hline
V&F&F\\
\hline
F&V&F\\
\hline
F&F&V\\
\hline
		\end{tabular}
	\end{center}
\caption[Tabla de verdad del operador lógico bicondicional.]{Tabla de verdad del operador lógico bicondicional. Para el caso del conector bicondicional es verdadero solo cuando las dos proposiciones tienen el mismo valor, es decir, o ambas son verdaderas o ambas son falsas.}
\end{table}
 
\begin{myexample}
Sean $p$ y $q$ dos proposiciones simples. Escriba el bicondicional entre ambas proposiciones.\\

 \noindent\textit{p:} ``El alumno aprueba el ramo.''\\
\textit{q:} ``El alumno promedia nota sobre $3,95$.''\\
\textit{$p\longleftrightarrow q$:} ``El alumno aprueba el ramo si y solo si promedia nota sobre $3,95$.''\\

\end{myexample} 
 
Entonces, para el caso de dos proposiciones unidas por los conectores lógicos se puede resumir en la siguiente tabla:

\begin{table}[h!]
	\begin{center}
		\begin{tabular}{||c|c||c|c|c|c|c||}
\hline
\hline
$p$&$q$&$p\wedge q$&$p\vee q$&$p\veebar q$&$p\rightarrow q$&$p\longleftrightarrow q$\\
\hline
V&V&V&V&F&V&V\\
\hline
V&F&F&V&V&F&F\\
\hline
F&V&F&V&V&V&F\\
\hline
F&F&F&F&F&V&V\\
\hline
\hline
		\end{tabular}
	\end{center}
\caption[Tabla de verdad resumen de todos los operadores lógicos.]{Tabla de verdad resumen de todos los operadores lógicos para dos proposiciones. Son los 5 operadores que forman proposiciones compuestas y en la tabla no se consideran las proposiciones negadas.}
\end{table}
 
%\begin{center}
%	\begin{tabular}{|c|c|c|}
%\hline
%$p$&$q$&\\
%\hline
%V&V&\\
%\hline
%V&F&\\
%\hline
%F&V&\\
%\hline
%F&F&\\
%\hline
%	\end{tabular}
%\end{center}

\subsection{Resultados de la tabla de verdad}
Se a visto en las secciones anteriores el resultado de la tabla de verdad con todas las combinaciones posibles, pero falta ver el caso más `real' y es cuando se utilizan varios conectivos lógicos. En estos casos, existe la posibilidad que el resultado de la tabla sean todos verdaderos, todos falsos o una mezcla de ellos.\\

\begin{mydef}
\textbf{Tautología:} Es siempre verdadera cualesquiera sean los valores de verdad de las proposiciones que la componen.
\end{mydef}

\begin{mydef}
\textbf{Contradicción:} Siempre es falsa, independiente de los valores de verdad de las proposiciones que la componen.
\end{mydef}

\begin{mydef}
\textbf{Contingencia:} No es tautología ni contradicción, es decir, hay resultados verdaderos y falsos en la tabla.
\end{mydef}


\begin{myexample}
Hacer la tabla de verdad de las siguientes proposiciones compuestas:
\end{myexample}
$1)$ $p\vee\sim p$.\\
\begin{center}
	\begin{tabular}{|c|c|c|}
\hline
$p$&$\sim p$&$p\vee\sim p$\\
\hline
V&F&V\\
\hline
F&V&V\\
\hline
	\end{tabular}
\end{center}
La proposición anterior es una tautología.\\

$2)$ $p\wedge \sim p$\\
\begin{center}
	\begin{tabular}{|c|c|c|}
\hline
$p$&$\sim p$& $p\wedge \sim p$\\
\hline
V&F&F\\
\hline
F&V&F\\
\hline
	\end{tabular}
\end{center}
La proposición anterior es una contradicción.\\

$3)$ $\sim p\vee q$
\begin{center}
	\begin{tabular}{|c|c|c|c|}
\hline
$p$&$q$&$\sim p$& $\sim p\vee q$\\
\hline
F&F&V&V\\
\hline
F&V&V&V\\
\hline
V&F&F&F\\
\hline
V&V&F&V\\
\hline
	\end{tabular}
\end{center}
La proposición anterior es una contingencia.\\

\begin{mydef}
\textbf{Lógicamente equivalentes}.
Dos proposiciones, $p$ y $q$, se consideran lógicamente equivalentes si sus tablas de verdad son idénticas o si su bicondicional es una tautología. Su notación es $p\Leftrightarrow q$ 
\end{mydef}

\begin{myexample}
Sean $p$ y $q$ dos proposiciones. Muestre que la tabla de verdad de $(p\rightarrow q)\Leftrightarrow (\sim q\rightarrow\sim p)$ es una tautología
\end{myexample}

\begin{center}
	\begin{tabular}{|c|c|c|c|c|c|c|}
\hline
$p$&$q$&$\sim p$& $\sim q$&$p\rightarrow q$&$\sim q\rightarrow\sim p$&$(p\rightarrow q)\Leftrightarrow (\sim q\rightarrow\sim p)$\\
\hline
V&V&F&F&V&V&V\\
\hline
V&F&F&V&F&F&V\\
\hline
F&V&V&F&V&V&V\\
\hline
F&F&V&V&V&V&V\\
\hline
	\end{tabular}
\end{center}

\subsection{Propiedades de las proposiciones}
Nuevamente mencionar que aquí, tanto a la izquierda como a la derecha de la fecha las tablas de verdad son idénticas. Esto permite que en una expresión lógica se pueda reemplazar una por otra.
\begin{itemize}
	\item Doble negación
		\subitem $\sim(\sim p)\Leftrightarrow p$
	\item Conmutatividad
		\subitem $p\wedge q\Leftrightarrow q\wedge p$
		\subitem $p\vee q\Leftrightarrow q\vee p$
		\subitem $(p\longleftrightarrow q)\Leftrightarrow(q\longleftrightarrow p)$
	\item Asociatividad
		\subitem $[(p\wedge q)\wedge r]\Leftrightarrow[p\wedge(q\wedge r)]$
		\subitem $[(p\vee q)\vee r]\Leftrightarrow[p\vee(q\vee r)]$
		\subitem $[(p\longleftrightarrow q)\longleftrightarrow r]\Leftrightarrow[p\longleftrightarrow(q\longleftrightarrow r)]$
	\item  Distributividad
		\subitem $[p\wedge(q\vee r)]\Leftrightarrow[(p\wedge q)\vee(p\wedge r)]$
		\subitem $[p\vee(q\wedge r)]\Leftrightarrow[(p\vee q)\wedge(p\vee r)]$
	\item Leyes de Morgan
		\subitem $\sim(p\wedge q) \Leftrightarrow (\sim p\vee\sim q)$
		\subitem $\sim(p\vee q)\Leftrightarrow(\sim p\wedge\sim q)$
	\item Idempotencia
		\subitem $(p\wedge p)\Leftrightarrow p$
		\subitem $(p\vee p)\Leftrightarrow p$
	\item $\sim(p\longrightarrow q)\Leftrightarrow (p\wedge\sim q)$
\end{itemize}
Se dice que $p$ implica lógicamente una proposición $q$, si $p\longrightarrow q$ es una tautología. Se denota con $p\Rightarrow q$ y se lee $p$ implica $q$.\\