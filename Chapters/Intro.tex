% A B S T R A C T
% ---------------

\begin{center}\textbf{Introducción}\end{center}

%Les doy la más cordial bienvenida a toda persona que se interesó a leer este texto, ya sea por obligación o por voluntad propia. El documento entrega nociones básicas de tópicos en matemáticas tales como: Lógica matemática, conjunto, funciones, cálculo vectorial y cálculo infinitesimal.\\
%
%Este texto está hecho para seguir una pauta personal del curso de Biomatemáticas para la facultad de medicina de la Universidad Católica de la Santísima Concepción y en ningún caso representa un texto formal o exigible para los alumnos de dicha asignatura.\\
%
%Cualquier error de tipeo, crítica constructiva o aporte que desee hacer contactar al correo imachuca@ucsc.cl 
%
%Este apunte se a echo en base a los libros guías \cite{Zill}, \cite{Larson} y \cite{Schaum}. Además se dan ejemplos externos que están referenciados.

Para poder entender cierto tópicos en el área de la salud y la biología en general se necesita crear un pensamiento a través de las matemáticas. En el presente libro se presentan los temas esenciales que necesita un estudiante o cualquier persona que quiera empezar a entender las matemáticas y la biología. Sin duda que en los tiempos actuales el profesional o docente no necesita calcular a diario, pero lo que si necesita es entender el funcionamiento de aparatos o procesos que la matemática explica.\\

El libro comienza con lógica matemática, que sin calcular nada, nos muestra como se confecciona las estructuras matemáticas. Sigue algo importantísimo como teoría de conjuntos, donde se profundiza en las operaciones que puedo hacer con ellos y técnicas de conteo para llevarlo a cabo en aplicaciones cuando se quiera clasificar y encontrar intersecciones de grupos. El calcular comienza con el tercer capítulo, el de los números reales. Se parte desde lo más sencillo y las implicancias que tienen sus propiedades. Más adelante viene el capítulo de desigualdades, que ayuda a entender de buena manera los intervalos de números. El siguiente capítulo es el central y el que forja el pensamiento que se debe tener siempre presente en esta y casi cualquier área. Es el capítulo de las funciones matemáticas y aquí aparecen los gráficos y sus variantes, que muestran sus traslaciones, deformaciones, crecimiento, decrecimiento y finalmente modelos que se pueden hacer con algunas de ellas. Luego, viene un capítulo un poco diferente, pero no menos importante. El cálculo vectorial nos muestra como nos podemos ubicar y contextualizar en ciertos problemas con matemática que viene desde la antigua Grecia, como la trigonometría. Finalmente, está el capítulo de cálculo infinitesimal con los límites, derivadas e integrales. Quizás la idea exacta no es tan familiar para todos, ya que las definiciones matemáticas no son del todo amigables, pero el trasfondo que tiene es de suma importancia. Los límites nos muestra como nos podemos acercar a cierto punto y conocer la función sin nunca llegar a el, las derivadas tienen una amplia aplicabilidad con el calculo de razones de cambio, aceleraciones que se utilizan en aparatos tecnológicos (modelar componentes electrónicos) hoy en día y las integrales con la idea que son el área bajo la curva nos permite tener un resultado para problemas que esta matemática sería casi imposible calcular.\\

Como idea final quiero dejar que el libro es una guía simplificada de tópicos en matemáticas y que se vuelva una herramienta accesible.

\cleardoublepage
%\newpage
